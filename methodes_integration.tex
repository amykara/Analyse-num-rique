# Create LaTeX file with examples for rectangle, trapezoid, Simpson methods
latex_content = r"""\documentclass{article}
\usepackage[utf8]{inputenc}
\usepackage{amsmath}
\title{Méthodes d'Intégration Numérique avec Exemples}
\author{}
\begin{document}
\maketitle

\section*{Fonction étudiée}
Soit la fonction : 
\[ f(x) = x^2 + 1 \]

Nous allons approximer l'intégrale : 
\[ \int_{0}^{6} f(x) \, dx \]

avec \( m = 6 \) subdivisions.

\section*{Méthode des Rectangles}
Formule :
\[ x_i = a + i \cdot h \]
\[ I \approx h \sum_{i=0}^{m-1} f(x_i) \]

Avec \( a = 0 \), \( b = 6 \), \( h = \frac{6}{6} = 1 \).

Exemple :
\begin{align*}
x_0 &= 0, & f(x_0) &= 1 \\
x_1 &= 1, & f(x_1) &= 2 \\
x_2 &= 2, & f(x_2) &= 5 \\
x_3 &= 3, & f(x_3) &= 10 \\
x_4 &= 4, & f(x_4) &= 17 \\
x_5 &= 5, & f(x_5) &= 26
\end{align*}
Somme :
\[ S = 1 + 2 + 5 + 10 + 17 + 26 = 61 \]
Résultat :
\[ I \approx 1 \cdot 61 = 61 \]

\section*{Méthode des Trapèzes}
Formule :
\[ I \approx \frac{h}{2} \left[f(a) + 2 \sum_{i=1}^{m-1} f(x_i) + f(b) \right] \]

Exemple :
\[ f(0) = 1, \quad f(6) = 37 \]
Somme intérieure :
\[ f(1) + f(2) + f(3) + f(4) + f(5) = 2 + 5 + 10 + 17 + 26 = 60 \]
Résultat :
\[ I \approx \frac{1}{2} \left[1 + 2 \cdot 60 + 37 \right] = \frac{1}{2} \cdot 158 = 79 \]

\section*{Méthode de Simpson}
Formule :
\[ I \approx \frac{h}{3} \sum_{i=0}^{n-1} \left[f(z_{2i}) + 4f(z_{2i+1}) + f(z_{2i+2}) \right] \]
Avec \( n = \frac{m}{2} = 3 \).

\begin{align*}
\text{Bloc } 0 &: (0, 1, 2) \Rightarrow 1 + 4 \cdot 2 + 5 = 14 \\
\text{Bloc } 1 &: (2, 3, 4) \Rightarrow 5 + 4 \cdot 10 + 17 = 62 \\
\text{Bloc } 2 &: (4, 5, 6) \Rightarrow 17 + 4 \cdot 26 + 37 = 158
\end{align*}

Somme :
\[ S = 14 + 62 + 158 = 234 \]
Résultat :
\[ I \approx \frac{1}{3} \cdot 234 = 78 \]

\end{document}
"""
with open("/mnt/data/methodes_integration.tex", "w") as f:
    f.write(latex_content)

